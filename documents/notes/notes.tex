\documentclass{article}

\usepackage[ngerman]{babel}
\usepackage[utf8]{inputenc}
\usepackage[T1]{fontenc}
\usepackage{hyperref}
\usepackage{csquotes}

\usepackage[
    backend=biber,
    style=apa,
    sortlocale=de_DE,
    natbib=true,
    url=false,
    doi=false,
    sortcites=true,
    sorting=nyt,
    isbn=false,
    hyperref=true,
    backref=false,
    giveninits=false,
    eprint=false]{biblatex}
\addbibresource{../references/bibliography.bib}

\title{Notizen zum Projekt Data Ethics}
\author{Latisha Soares de Carvalho}
\date{\today}

\begin{document}
\maketitle

\abstract{
    Dieses Dokument ist eine Sammlung von Notizen zu dem Projekt. Die Struktur innerhalb des
    Projektes ist gleich ausgelegt wie in der Hauptarbeit, somit kann hier einfach geschrieben
    werden, und die Teile die man verwenden möchte, kann man direkt in die Hauptdatei ziehen.
}



\tableofcontents
\section {Einleitung}                            
    
    Die Einführung in das KI-Training beginnt mit der Eingabe von Daten in ein Computersystem.
    Mithilfe von Techniken des maschinellen Lernens , einschließlich Deep Learning, können Daten analysiert und bessere Vorhersagen getroffen werden
    Dieser Trainingsprozess ermöglicht es der Software, Merkmale in Daten zu erkennen und sich zu verbessern.
    
    Es gibt zwei Hauptmethoden des KI-Trainings: überwachtes und unüberwachtes Lernen.
    Beim überwachten Lernen werden beschriftete Daten zum Trainieren verwendet, während beim unbeaufsichtigten Lernen das Modell von Mustern in unbeschrifteten Daten verwendet wird.
    
    Der nächste Schritt des Trainingsprozesses ist die Validierung. Dabei wird die Leistung des Modells anhand unabhängiger Modelle bewertet, um festzustellen, ob das Training angepasst werden muss.
    Dazu gehört auch ein vorzeitiger Abbruch der Ausbildung, wenn wesentliche Verbesserungen nicht mehr zu erwarten sind.

    Nach der -Validierung folgt der Test, bei dem das Modell auf unstrukturierte Daten angewendet wird, um seine Echtheit in der realen Welt zu überprüfen.
    Mögliche Probleme wie Über- oder Unteranpassung müssen berücksichtigt werden und das Modell sollte bei Bedarf angepasst werden.

    Beim KI-Training müssen mehrere Faktoren berücksichtigt werden , darunter die Qualität der Daten, die erforderliche Software sowie die Ressourcen und Verfügbarkeit.
    Eine sorgfältige Planung und Ausführung des Prozesses ist für die Entwicklung einer guten KI-Modells wichtig.

    %% Quelle: https://www.clickworker.de/kunden-blog/kuenstliche-intelligenz-systeme-trainieren/

\section {Fragestellung} 
    Wer trägt die Verantwortung, wenn ein KI-System fehlerhafte oder unethische Entscheidungen trifft?

\section {Einführung zu meinem Thema}
In einer Welt mit KI stellt sich eine Frage: Wer ist verantwortlich, 
wenn ein KI-System falsche oder unethische Entscheidungen trifft? 
Stell dir vor, -> Bsp: man geht zum Arzt und wird falsch diagnostiziert, da er ein KI-system verwendet.
 Oder selbstfahrende Autos.
 Wer hat die Verantwortung ? Hersteller? Arzt? 
 Wer ist verantwortlich, wenn das Auto eine unerwartete
 Entscheidung trifft? Diese Beispiele zeigen, dass wir klare Regeln brauchen, um festzulegen, wer verantwortlich ist, wenn KI-Systeme Probleme haben. 
 Die Beispiele verdeutlichen, wie wichtig es ist

\section {Was ist KI}
- keine offizielle Beschreibung
- menschliches Verhalten
- Problemlöser
%% Quelle: https://www.t-online.de/digital/internet/id_100310396/wie-funktioniert-ki-kuenstliche-intelligenz-einfach-erklaert.html

\section {Haftungsverpflichtung}
- Die Haftung bei Schäden hängt von Faktoren ab
- rechtliche Herausforderungen
- Hersteller können haften durch Produkthaftgesetz und Produzentenhaftung
- autonome Autos komplikationen
3 Sichten -> Hersteller, Nutzer, die gemeinsame Verantwortung
spezieller Fall: autonome Systeme (Autos, die alleine fahren können)

%% Quelle: https://www.srd-rechtsanwaelte.de/blog/kuenstliche-intelligenz-haftung/  https://legal.pwc.de/de/news/fachbeitraege/haftung-fuer-schaeden-durch-kuenstliche-intelligenz-der-richtlinienvorschlag-der-eu-kommission

 \section {Quellen}
 \url {https://www.clickworker.de/kunden-blog/kuenstliche-intelligenz-systeme-trainieren/}
 \url{https://www.t-online.de/digital/internet/id_100310396/wie-funktioniert-ki-kuenstliche-intelligenz-einfach-erklaert.html}
 \url {https://www.srd-rechtsanwaelte.de/blog/kuenstliche-intelligenz-haftung/}
 \url {https://legal.pwc.de/de/news/fachbeitraege/haftung-fuer-schaeden-durch-kuenstliche-intelligenz-der-richtlinienvorschlag-der-eu-kommission}



\end{document}
