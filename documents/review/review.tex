\documentclass{article}

\usepackage[ngerman]{babel}
\usepackage[utf8]{inputenc}
\usepackage[T1]{fontenc}
\usepackage{hyperref}
\usepackage{csquotes}

\usepackage[
    backend=biber,
    style=apa,
    sortlocale=de_DE,
    natbib=true,
    url=false,
    doi=false,
    sortcites=true,
    sorting=nyt,
    isbn=false,
    hyperref=true,
    backref=false,
    giveninits=false,
    eprint=false]{biblatex}
\addbibresource{../references/bibliography.bib}

\title{Review des Papers "Ethik im Umgang mit Daten" von <Marina Brun>}
\author{Latisha Soares de Carvalho}
\date{\today}

\begin{document}
\maketitle

\abstract{
    Dies ist ein Review der Arbeit zum Thema Ethik im Umgang mit Daten von Marina Brun.
}

\tableofcontents


\section{Einleitung}
Der Titel des Papers lautet "Ethik im Umgang mit Daten", geschrieben von Marina Brun.
In dem folgendem Thema spricht Marina über die Herausforderungen und Risiken, die die KI im Thema Datenschutz und Privatsphäre mitsichbringt.
In diesem Review werde ich das Paper, welches sie geschrieben hat, analysieren und dementsprechend bewerten.
Um mein Review zu strukturieren werde ich folgende Themen behandeln: Zusammenfassung der Arbeit, Analyse, Verbesserungsvorschläge und Gesamtbewertung.

\section{Zusammenfassung der Arbeit}
Marina spricht viele verschiedene Themen in ihrer Arbeit an, die hauptsächlich die Datenschutzproblemen erklären, die durch
die Benutzung von KI-Systemen verursacht werden. Dabei werden potenzielle Gefahren für personenbeszogene Daten, wie z.B. Datenmissbrauch,
Cyberangriffe, Identitätsdiebstahl und unerwünschte Datenaufzeichnungen durch digitale Assistenten ins Zentrum gebracht.

\section{Analyse}
\subsection{Fragestellung}
Marinas Fragestellung wird direkt deutlich und lautet "Welche Bedenken gibt es hinsichtlich des Datenschutzes und der Privatsphäre
in KI-Anwendungen?" Im Verlauf des Papers wird diese Fragestellung durch Argumenten und Themen beantwortet. Die Frage ist aus meiner Hinsicht gut 
und aktuell. Der Schutz der Daten, genauso wie die Privatsphäre, ist immer wichtiger. Insbesondere wird es in einer Generation , indem ein Leben mit KI immer mehr verbreitet wird.
Marina hat es geschafft die Frage umfassend zu beantworten.

\subsection{Struktur}
Die deutliche Struktur und die bisherige schon genannte umfassende Behandlung des Themas, bietet die Gelegenheit für die Lesern, welche keine Vorkenntnisse haben, solche Probleme zu verstehen.
Marina packt ihre Arbeit in ein paar Untertiteln, was dazu führt es dem Leser verständlicher zu machen. Somit ist die Struktur der Arbeit logisch aufgebaut.

\section{Verbesserungsvorschläge}
Die Arbeit an sich ist sehr gut, was ich immer wieder in anderen Abschnitten deutlich mache.Etwas anderes was ich gut finde, ist dass Marina ein Bild hinzugefügt hat, um ihren Text nochmals, diesmal bildlich, genauer zu erklären. Was man aber vielleicht hinzufügen könnte, wären alltägliche Beispiele in denen Datenschutzverletzungen vorkommen. Dies hilft
dem Leser die Riskien noch genauer wahrzunehmen. Ein anderer Verbesserungsvorschlag wäre, dass man die Fachbegriffe kurz erklären könnte wie z.B. Cyberangriffe, da es womöglich nicht für jeden klar ist, was dies bedeutet. 
\section{Gesamtbewertung}
Zu guter letzt werde ich eine Gesamtbewertung abgeben. Ich fand das Paper stellte insgesamt eine gute, tiefgründige Nachforschung dar. Die Arbeit ist sehr gut verfasst, geanuso wie die angewendete Sprache. Beim Durchlesen 
des Textes konnte man ohne Bemühungen weiterverfolgen, um was es explizit geht. Es bietet eine gute Grundlage für die Weiterentwicklung der Arbeit in diesem Apsekt.  

\printbibliography

\end{document}
